%
% File naaclhlt2012.tex
%

\documentclass[11pt,letterpaper]{article}
\usepackage{times}
\usepackage{latexsym}
\usepackage{amsmath}
\usepackage{color}

\title{A Summary of the Dialog State Tracking Challenge Results}

\author{Wencan Luo\\
	  }

\date{September 19, 2013}

\begin{document}
\maketitle

\section{Results}
9 teams entered the DSTC, submitting a total of 27 trackers.

There are three schedules for determining which turns to include in each evaluation.

{\bf Schedule 1}: Include all turns.

{\bf Schedule 2}: Include a turn for a given concept only if that concept either appears on the SLU N-Best list in that turn, or if the system�s action references that concept in that turn.

{\bf Schedule 3}: Include only the turn before the system starts over from the beginning, and the last turn of the dialog.

It evluated in 5 different data sets. The number of turns under different schedules are shown in Table \ref{table:counts}

\begin{table*}[!htb]
\centering
\begin{tabular}{c|c|c|c|c|c|c}
&test1&test2&test3&test4&train2&train3\\\hline
schedule1&90765&97713&119376&42786&41211&40131\\
schedule2&27536&23028&12017&7348&12493&3697\\
schedule3&5784&3023&3108&2217&2589&1119\\
\end{tabular}
\caption{Number of turns under different schedules and different data sets}
\label{table:counts}
\end{table*}

\begin{table}[!htb]
\centering
\begin{footnotesize}
\begin{tabular}{c|c|c|c|c|c|c|c|c|c|c}
team&entry&slot&schedule&metric&test1&test2&test3&test4&train2&train3\\\hline
baseline&entry0&all&schedule1&accuracy&0.7748&0.7928&0.9178&0.8233&0.7523&0.9313\\\hline
team0&entry0&all&schedule1&accuracy&0.8162&0.7971&0.7720&0.8434&0.7837&0.7207\\
team0&entry1&all&schedule1&accuracy&0.7748&0.7928&0.9178&0.8233&0.7523&0.9313\\\hline
team1&entry1&all&schedule1&accuracy&0.8853&0.8465&0.9230&0.8479&0.8664&0.9459\\\hline
team2&entry1&all&schedule1&accuracy&0.8674&0.8844&0.9435&0.8351&0.8461&0.9451\\
team2&entry2&all&schedule1&accuracy&0.8673&0.8832&0.9432&0.8346&0.8454&0.9348\\\hline
team3&entry1&all&schedule1&accuracy&0.8585&0.8752&0.9458&0.7279&0.8462&0.9440\\
team3&entry2&all&schedule1&accuracy&0.8661&0.8687&0.9315&0.8631&0.8469&0.9274\\
team3&entry3&all&schedule1&accuracy&0.8814&0.8836&&&0.8612&\\\hline
team4&entry1&all&schedule1&accuracy&0.7975&0.8445&0.9344&0.6475&0.7947&0.9540\\\hline
team5&entry1&all&schedule1&accuracy&0.8576&0.8389&0.9304&0.8500&0.8886&0.9539\\
team5&entry2&all&schedule1&accuracy&0.8577&0.8438&0.9222&0.8681&0.8920&0.9509\\
team5&entry3&all&schedule1&accuracy&0.7301&0.7677&0.9013&&0.8850&0.9429\\
team5&entry4&all&schedule1&accuracy&0.8530&0.8563&0.9118&&0.8887&0.9502\\
team5&entry5&all&schedule1&accuracy&0.8649&0.8848&0.9045&&0.8857&0.9537\\\hline
team6&entry1&all&schedule1&accuracy&0.9115&{$\color{red}0.9240^*$}&0.8424&0.8673&{$\color{red}0.9085^*$}&0.8855\\
team6&entry2&all&schedule1&accuracy&0.8874&0.9047&0.9481&0.8428&0.8808&{$\color{red}0.9700^*$}\\
team6&entry3&all&schedule1&accuracy&0.9170&0.9202&0.9386&0.8678&0.9045&0.9656\\
team6&entry4&all&schedule1&accuracy&{$\color{red}0.9171^*$}&0.9221&0.9408&0.8672&0.9033&0.9660\\
team6&entry5&all&schedule1&accuracy&0.8888&0.9043&0.9486&0.8457&0.8784&0.9695\\\hline
team7&entry1&all&schedule1&accuracy&0.8440&0.8548&0.9224&0.7657&0.8129&0.9333\\\hline
team8&entry1&all&schedule1&accuracy&0.8283&0.8155&0.8067&0.8289&0.8380&0.8141\\
team8&entry2&all&schedule1&accuracy&0.8200&0.7982&0.8067&0.8090&0.8571&0.8117\\
team8&entry3&all&schedule1&accuracy&0.5495&0.6606&0.9214&0.7870&0.6118&0.9523\\
team8&entry4&all&schedule1&accuracy&0.7800&0.8088&0.9131&0.7973&0.7833&0.9383\\
team8&entry5&all&schedule1&accuracy&0.7684&0.8087&0.9102&0.8128&0.7662&0.9383\\\hline
team9&entry1&all&schedule1&accuracy&0.8770&0.8725&0.9441&0.8690&0.8842&0.9679\\
team9&entry2&all&schedule1&accuracy&0.8732&0.8711&0.9437&0.8701&0.8805&0.9662\\
team9&entry3&all&schedule1&accuracy&0.8821&0.8825&0.9479&0.8466&0.8822&0.9644\\
team9&entry4&all&schedule1&accuracy&0.8798&0.8844&{$\color{red}0.9487^*$}&0.8415&0.8843&0.9627\\
team9&entry5&all&schedule1&accuracy&0.8286&0.8276&0.8918&{$\color{red}0.8802^*$}&0.8427&0.9455\\
\end{tabular}
\end{footnotesize}
\caption{Accuracy of all entries on schedule 1}
\label{table:s1}
\end{table}

\begin{table}[!htb]
\centering
\begin{footnotesize}
\begin{tabular}{c|c|c|c|c|c|c|c|c|c|c}
team&entry&slot&schedule&metric&test1&test2&test3&test4&train2&train3\\
baseline&entry0&all&schedule2&accuracy&0.6020&0.4905&0.6202&0.5841&0.5485&0.6641\\
team0&entry0&all&schedule2&accuracy&0.7056&0.5267&0.2590&0.6585&0.6097&0.2440\\
team0&entry1&all&schedule2&accuracy&0.6020&0.4905&0.6202&0.5841&0.5485&0.6641\\
team1&entry1&all&schedule2&accuracy&0.7686&0.6011&0.5948&0.6787&0.7347&0.6760\\
team2&entry1&all&schedule2&accuracy&0.7167&0.6649&0.6424&0.6196&0.6937&0.6960\\
team2&entry2&all&schedule2&accuracy&0.7165&0.6617&0.6418&0.6183&0.6922&0.6889\\
team3&entry1&all&schedule2&accuracy&0.7221&0.6564&0.6453&0.5328&0.7044&0.7235\\
team3&entry2&all&schedule2&accuracy&0.7186&0.6135&0.5629&0.6826&0.6853&0.7062\\
team3&entry3&all&schedule2&accuracy&0.7643&0.6735&&&&0.6394\\
team4&entry1&all&schedule2&accuracy&0.5803&0.5428&0.5699&0.2654&0.5975&0.6668\\
team5&entry1&all&schedule2&accuracy&0.7654&0.6165&0.6639&0.6448&0.8065&0.7812\\
team5&entry2&all&schedule2&accuracy&0.7650&0.6255&0.6405&0.6956&0.8097&{$\color{red}0.7841^*$}\\
team5&entry3&all&schedule2&accuracy&0.4688&0.3813&0.5689&&0.7941&0.7506\\
team5&entry4&all&schedule2&accuracy&0.7557&0.6312&0.5897&&0.8061&0.7544\\
team5&entry5&all&schedule2&accuracy&0.7757&0.6958&0.5791&&0.7972&0.7614\\
team6&entry1&all&schedule2&accuracy&0.8172&{$\color{red}0.7784^*$}&0.4367&0.6967&{$\color{red}0.8155^*$}&0.5926\\
team6&entry2&all&schedule2&accuracy&0.7593&0.7206&0.6740&0.6293&0.7608&0.7833\\
team6&entry3&all&schedule2&accuracy&0.8220&0.7586&0.6321&0.6918&0.8076&0.7601\\
team6&entry4&all&schedule2&accuracy&{$\color{red}0.8223^*$}&0.7619&0.6418&0.6885&0.8052&0.7601\\
team6&entry5&all&schedule2&accuracy&0.7630&0.7190&{$\color{red}0.6799^*$}&0.6358&0.7566&0.7801\\
team7&entry1&all&schedule2&accuracy&0.6880&0.5669&0.5399&0.5078&0.6151&0.6494\\
team8&entry1&all&schedule2&accuracy&0.7226&0.5733&0.3155&0.6260&0.7212&0.4444\\
team8&entry2&all&schedule2&accuracy&0.7048&0.5386&0.3031&0.5870&0.7559&0.4252\\
team8&entry3&all&schedule2&accuracy&0.2365&0.1884&0.6162&0.5275&0.3137&0.7698\\
team8&entry4&all&schedule2&accuracy&0.6283&0.5224&0.6027&0.5378&0.6211&0.7560\\
team8&entry5&all&schedule2&accuracy&0.6097&0.5193&0.6024&0.5542&0.5902&0.7560\\
team9&entry1&all&schedule2&accuracy&0.7824&0.6736&0.6568&0.7050&0.7772&0.7771\\
team9&entry2&all&schedule2&accuracy&0.7776&0.6720&0.6456&0.7030&0.7733&0.7590\\
team9&entry3&all&schedule2&accuracy&0.7848&0.6943&0.6413&0.6498&0.7760&0.7222\\
team9&entry4&all&schedule2&accuracy&0.7820&0.6999&0.6381&0.6399&0.7778&0.7108\\
team9&entry5&all&schedule2&accuracy&0.7265&0.5887&0.4837&{$\color{red}0.7368^*$}&0.7091&0.6746\\
\end{tabular}
\end{footnotesize}
\caption{Accuracy of all entries on schedule 2}
\label{table:s2}
\end{table}

\begin{table}[!htb]
\centering
\begin{footnotesize}
\begin{tabular}{c|c|c|c|c|c|c|c|c|c|c}
team&entry&slot&schedule&metric&test1&test2&test3&test4&train2&train3\\
baseline&entry0&all&schedule2&accuracy&0.5982&0.4869&0.7033&0.6396&0.5647&0.7480\\
team0&entry0&all&schedule3&accuracy&0.6255&0.4790&0.1200&0.6234&0.5674&0.0822\\
team0&entry1&all&schedule3&accuracy&0.5982&0.4869&0.7033&0.6396&0.5647&0.7480\\
team1&entry1&all&schedule3&accuracy&0.7818&0.6381&0.7095&0.6261&0.7482&0.8195\\
team2&entry1&all&schedule3&accuracy&0.7768&0.7641&0.7973&0.6576&0.7594&0.8320\\
team2&entry2&all&schedule3&accuracy&0.7765&0.7605&0.7947&0.6558&0.7578&0.8186\\
team3&entry1&all&schedule3&accuracy&0.7476&0.7228&0.7938&0.3112&0.7300&0.8213\\
team3&entry2&all&schedule3&accuracy&0.7590&0.7152&0.7590&0.7028&0.7319&0.7945\\
team3&entry3&all&schedule3&accuracy&0.7877&0.7469&&&0.7509&\\
team4&entry1&all&schedule3&accuracy&0.6912&0.6705&0.7722&0.2950&0.6987&0.8463\\
team5&entry1&all&schedule3&accuracy&0.7284&0.6166&0.7597&0.6748&0.7837&0.8570\\
team5&entry2&all&schedule3&accuracy&0.7294&0.6325&0.7301&0.7100&0.7937&0.8472\\
team5&entry3&all&schedule3&accuracy&0.5820&0.4803&0.6409&&0.7791&0.8436\\
team5&entry4&all&schedule3&accuracy&0.6990&0.6414&0.6831&&0.7791&0.8391\\
team5&entry5&all&schedule3&accuracy&0.7438&0.7397&0.6577&&0.7806&0.8579\\
team6&entry1&all&schedule3&accuracy&0.8437&{$\color{red}0.8544^*$}&0.4196&0.6775&0.8304&0.6318\\
team6&entry2&all&schedule3&accuracy&0.8276&0.8131&0.8214&0.6852&0.8080&0.8999\\
team6&entry3&all&schedule3&accuracy&0.8620&0.8445&0.7857&0.7221&{$\color{red}9240^*$}&0.8432\\
team6&entry4&all&schedule3&accuracy&{$\color{red}0.8622^*$}&0.8498&0.7944&0.7212&0.8428&0.8803\\
team6&entry5&all&schedule3&accuracy&0.8276&0.8121&0.8234&0.6901&0.8073&{$\color{red}0.9008^*$}\\
team7&entry1&all&schedule3&accuracy&0.6964&0.6891&0.6239&0.5016&0.6628&0.7042\\
team8&entry1&all&schedule3&accuracy&0.6888&0.5577&0.2407&0.6216&0.6914&0.3718\\
team8&entry2&all&schedule3&accuracy&0.6739&0.5147&0.2416&0.5918&0.7563&0.3753\\
team8&entry3&all&schedule3&accuracy&0.3119&0.2455&0.7156&0.5498&0.3774&0.8570\\
team8&entry4&all&schedule3&accuracy&0.6129&0.5471&0.6856&0.5760&0.6099&0.8329\\
team8&entry5&all&schedule3&accuracy&0.5877&0.5468&0.6766&0.6139&0.5759&0.8329\\
team9&entry1&all&schedule3&accuracy&0.7827&0.7033&0.8034&0.7262&0.7806&0.8928\\
team9&entry2&all&schedule3&accuracy&0.7780&0.7049&0.8057&0.7258&0.7783&0.8865\\
team9&entry3&all&schedule3&accuracy&0.7894&0.7321&0.8240&0.6951&0.7771&0.8829\\
team9&entry4&all&schedule3&accuracy&0.7872&0.7476&{$\color{red}0.8279^*$}&0.6816&0.7876&0.8758\\
team9&entry5&all&schedule3&accuracy&0.6712&0.5703&0.6116&{$\color{red}0.7406^*$}&0.6956&0.8213\\
\end{tabular}
\end{footnotesize}
\caption{Accuracy of all entries on schedule 3}
\label{table:s3}
\end{table}

\section{Features}

The features are shown in Table \ref{table:feature}.
\begin{table}[!htb]
\centering
\begin{tabular}{c|l}\\
Team&Features\\\hline
Henderson et al.&SLU score\\
&Rank score\\
&Affirm score\\
&Negate score\\
&Go back score\\
&Implicit score\\
&User act type\\
&Machine act type\\
&Cant help\\
&Slot confirmed\\
&Slot requested\\
&Slot informed\\\hline
Ren et al.&Pairwise-slots of the same rank\\
&Pairwise-slots with identical value\\
&SLU score and rank of slot\\
&Dialog history (grounding information)\\
&Count of slots with identical value\\
&Domain-specific features\\
&Baseline Tracker\\\hline
Metallinou et al.&rank of the current SLU result\\
&the SLU result confidence score(s)\\
&the difference between the current hypothesis score and the best\\
&the number of times an SLU result has been observed before\\
&the number of times an SLU result has been observed before at a specific rank\\
&the sum and average of confidence scores\\
&the number of possible past user negations or confirmations \\\hline
Lee 1&$informs_k(y,x_1^t)$\\
&$affirm_k(y,x_1^t)$\\
&$max\_score_k(y,x_1^t)$\\
&$acc\_score(y,x_1^t)$\\
&$pbm\_score(y,x_1^t)$\\
&$prior_k(y,x_1^t)$\\
&$canthelpk(y,x_1^t)$\\
&$bias(y,x_1^t)$\\
&$bias_{none}(y,x_1^t)$\\
\end{tabular}\\

\caption{Features}
\label{table:feature}
\end{table}

\section{TODO}

\begin{itemize}
	\item Get the Upper Bound of N-Best
\end{itemize}

\end{document}
