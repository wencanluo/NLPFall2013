%
% File naaclhlt2012.tex
%

\documentclass[11pt,letterpaper]{article}
\usepackage{naaclhlt2012}
\usepackage{times}
\usepackage{latexsym}
\setlength\titlebox{6.5cm}    % Expanding the titlebox

\title{Semantic Rescoring Framework of Spoken Language Understandong}

\author{Wencan Luo\\
	    Department of Computer Science\\
	    University of Pittsburgh\\
	    PA 15260, USA\\
	    {\tt wencan@cs.pitt.edu}
	  }

\date{September 4, 2013}

\begin{document}
\maketitle
\begin{abstract}
  
\end{abstract}

\section{Introduction}
Let a computer understand what people say is a long term goal. However, speech recognition (SR) is not perfect in current stage. 
In fact, compared to manual trascriptions, SR errors have a significant decrease in many NLP Tasks, such as Quesition Answering \cite{Turmo:2007}, Speaker Role Recognition \cite{Garg:2008}, Natural Language Understanding (NLU) \cite{Raymond:2007}, etc.

\section{Related Work}
\subsection{NLU}

NLU is a well-study field and many techniques have been proposed to improve the performance. For example, Eun at al.\shortcite{Eun:2005} showed that combining several different classifiers promoted the performance of natural language understanding. 
 
Besides, Both generative and discriminative models work pretty well for spoken language understanding \cite{Raymond:2007}.
 
However, majority of people work on human-transcribed text but rather than directly on speech recognition results. However, without ignoring the speech recognition errors, the performance is expected to decrease a lot. 

\subsection{Rescoring N-Best}

\section{Methodology}

\section{Timeline}

\noindent \emph{Sep 09 - Sep 22}
\begin{itemize}
  \item survey the related work regarding role recognition
  \item understanding the data, know how to extract and use the data
\end{itemize}

\noindent \emph{Sep 23 -  Oct 20}
\begin{itemize}
  \item implement the method in \cite{Garg:2008} using the manual transcription, the lexical model will be used as the local model
  \item do Speech Recognition (SR)
  \item run the local model on SR results
\end{itemize}

\noindent \emph{Oct 21 - Nov 9}
\begin{itemize}
  \item implement ILP global model, using the manual speaker segmentation
\end{itemize}

\noindent \emph{Nov 10 - Dec 12}
\begin{itemize}
  \item propose a model without the manual speaker segmentation
  \item try other global model such as Bayes network, improved social network
\end{itemize}

\section*{Acknowledgments}

Do not number the acknowledgment section.

\begin{thebibliography}{}
\bibitem[\protect\citename{Turmo et al.}2007]{Turmo:2007}
J. Turmo, P. Comas, C. Ayache, D. Mostefa, S. Rosset, L. Lamel.
\newblock 2007.
\newblock {\em Overview of QAST 2007}.
\newblock In Working Notes of CLEF 2007 Workshop, Budapest, Hungary.

\bibitem[\protect\citename{Garg et al.}2008]{Garg:2008}
N. Garg, S. Favre, H. Salamin, D. Hakkani-Tur, and A. Vinciarelli. 
\newblock 2008. 
\newblock {\em Role recognition for meeting participants:an approach based on lexical information and social network analysis}.
\newblock Proceedings ACM International Conference on Multimedia, 693-696.

\bibitem[\protect\citename{Raymond and Riccardi}2007]{Raymond:2007} 
C. Raymond and G. Riccardi. 
\newblock 2005. 
\newblock {\em Generative and discriminative algorithms for spoken language understanding.} 
\newblock In Interspeech, pp. 1605–1608, Antwerp, Belgium, Aug. 2007. 

\end{thebibliography}

\end{document}
