%
% File naaclhlt2012.tex
%

\documentclass[11pt,letterpaper]{article}
\usepackage{naaclhlt2012}
\usepackage{times}
\usepackage{latexsym}
\setlength\titlebox{6.5cm}    % Expanding the titlebox

\title{CS2002 Proposal: Uncertainty Prediction on Mars Exploration Rover (MER)
 scientist dataset and the student engineering team dataset}

\author{Wencan Luo (advised by Diane Litman)\\
	    Department of Computer Science\\
	    University of Pittsburgh\\
	    PA 15260, USA\\
	    {\tt wencan@cs.pitt.edu}
	  }

\date{January 11, 2012}

\begin{document}
\maketitle
\begin{abstract}
  This document is the proposal for the CS2001 project: Uncertainty Prediction on Mars Exploration Rover (MER)
 scientist dataset and the student engineering team dataset. The main goal for this 
 project is to retrieve the uncertainty utterance as much as possible in these two date sets. To do so, supervised 
 and unsupervised methods are both to try.
\end{abstract}

\section{Introduction}
Participants in dialog coordinate the presentation and acceptance of their utterances to establish, 
maintain and confirm mutual understanding, which called grounding. However, uncertainty are always involved between the 
speaker and hearer. Human beings can adjust their dialog by detecting the uncertainty to make sure they understood each
other. Then, when comes to computer dialog system, detecting such state enables the system to respond more appropriately. 
For example, in tutoring dialog system, the tutor can adjust the dialog content according 
to the uncertainty level dynamically. In fact, \cite{Forbes-Riley:2009} shows that 
adapting to student uncertainty did improve the overall performance
 in term of user satisfaction and learning efficiency.

\section{Problem}
There are two datasets: the Mars Exploration Rover (MER) scientist dataset, including 12,331 utterances,
and the student engineering team datasets, including 16772 utterances. 
Audio/video are available for both datasets. The student engineering teams are natural teams working on 
their semester-long projects; the MER scientists are evaluating data downloaded from the rover, discussing their 
work process, and/or making plans for the rovers.

The MER scientists come from a large team of about 100+ scientists/faculty, graduate students, and  
technicians. At any one time, conversations are between 2-10 people. The engineering dataset involves student teams
 of conversations of 2-6 individuals.
 
Among the MER data, 1641 of all the 12,331 (13.3\%) utterances are annotated as uncertainty by human; 
however, for engineering dataset, only 1558 utterances are annotated and 221 of which are uncertainty (14.2\%).
The first research problem is to build a model to automatically predict the uncertainty.

From the above description, there are some points that are different between the two data sets. 
Firstly, the occupations of subjects are different. One is diverse including scientists, graduate students and technicians; 
the other is all students; secondly, the number of individuals in each dialog is different; 
thirdly, the topics in the dialog are different.

Therefore, we expect the uncertainty models in these different data sets are also different. Then, the research question 
is what are the difference.

\section{Goals}
The goal for this project is to design a model that has a high recall, which means to retrieve the uncertainty utterance
as much as possible. There may be many noises in the raw data sets, even annotated by human. Thus, this model serves as 
a pre-processing stage rather than a final model. In advance, the false negative 
instances given by the model can be rechecked by human in the further step. 

\section{Related Work}
\subsection{Definition of Uncertainty}
\label{sect:related_work}

As indicated in \cite{Barry_Shieber:2011}, there are two different meaning of uncertainty. 
The one refers to how certain a person sounds, the perceived level of certainty; the other one 
is about the internal level of certainty, how certain speakers actually are.

In this paper, the later one is concerned.

\subsection{Uncertainty Prediction}
There are several literatures that try to predict uncertainty in dialog, the most recent ones 
among which are \cite{Barry_Shieber:2011} and \cite{Forbes:2011}. Some common prosodic 
features are both used in these two papers such as pitch, intensity and temporal features. 
In fact, these prosodic features have also been used by many other research on speech affection detection  
\cite{Breazeal:2002}, \cite{Dellaert:1996}, \cite{Lee:2001}.

Besides the prosodic features, lexical features are also popular to predict uncertainty. 
Commonly-used lexical features are unigram, word length, and part of speech. 
In addition, utterance's familiarity and position are also included in \cite{Barry_Shieber:2011}.
For tutoring dialog system, discourse features like tutor goal, problem, turn are considered in \cite{Forbes:2011}.
Other features like the speaker's gender are also benefiting.

In this paper, we want to develop more features on lexical.

\section{Methodology}
\subsection{Rule-based Method}
According to the guideline to annotate the corpus, we can write rules to predict the uncertainty as a baseline model.

Followings are some of the rules.

The clause below shows no evidence of uncertainty:
\begin{itemize}
  \item I/It can/would/will/won��t ��
  \item I am/It is��
\end{itemize}

The clause below shows some evidence of uncertainty:
\begin{itemize}
  \item you cannot tell whether X
  \item some kind/sort of X
  \item something like this
  \item as far as I/we know
  \item somehow
  \item I didn��t know X
  \item I guess/think/believe/feel/hope
  \item may/might/could
  \item probably/possibly/maybe
  \item sort of/kind of
  \item worried that
  \item really
  \item not really
  \item he��s somewhere in space out there
  \item that can vary
  \item more or less
\end{itemize}

\subsection{Machine Learning Method}
As each utterance is either uncertainty or certainty (neutral one is not considered in this paper), so it can be treated as a binary classification problem.
Therefore, machine learning methods can be applied. The key points here are the features set. 
Inspired by relating and our previous work, I want to try the following features, including lexical or discourse.

The features are shown in Table~\ref{table:feature}.

In particular, for the unigram, bigram or trigram features, not all the lexical words are included,
but only the words that appear more than 14\% in the uncertainty utterances compared to certainty ones. In this way, 
features number can be reduced.

\begin{table}
\begin{center}
\noindent \begin{tabular}{ l | l | l }
	\hline 
	lexical & unigram &  \\
		& bigram &  \\
		& trigram &  \\
		& overlap &  \\
		& utterance length &  \\
		& subject &  I, you, other \\
	\hline 
\end{tabular}
\end{center}
\caption{\label{table:feature} Feature Set. }
\end{table}

\section{Experiments}
Model that has the maximal recall will be the final one. Therefore, we need to compare models
under different feature sets.

In addition, we also want to compare the performance between different data sets.

\section{Timeline}

\noindent \emph{Jan 12 - Jan 20}
\begin{itemize}
  \item understanding the data, know how to extract and use the data
  \item survey the related work regarding uncertainty prediction based on text
\end{itemize}

\noindent \emph{Jan 21 - Feb 8}
\begin{itemize}
  \item coding the rule-based methods
  \item extract the unigram, bigram and trigram features
  \item prelim results based this simple lexical features using Weka
  \item present the prelim results
\end{itemize}

\noindent \emph{Feb 9 - March 9}
\begin{itemize}
  \item based on the suggestions, improve the model
  \item develop more lexical features
  \item extract other features, such as syntax, semantic
\end{itemize}

\noindent \emph{March 10 - April 20}
\begin{itemize}
  \item improve the model based on new features
  \item anaylze the confusion matrix of the model and improve the model based on the results
  \item write the workshop-like paper
\end{itemize}

\section*{Acknowledgments}

Do not number the acknowledgment section.

\begin{thebibliography}{}
\bibitem[\protect\citename{Drummond and Litman}2011]{Drummond-Litman:2011}
Drummond, Joanna and Litman, Diane.
\newblock 2011.
\newblock {\em Examining the Impacts of Dialogue Content and System Automation on Affect Models in a Spoken Tutorial Dialogue System}.
\newblock Proceedings of the SIGDIAL 2011 Conference, Portland, Oregon.

\bibitem[\protect\citename{Forbes et al.}2011]{Forbes:2011}
Drummond, Joanna and Litman, Diane.
\newblock 2011.
\newblock {\em Benefits and challenges of real-time uncertainty detection and adaptation in a spoken dialogue computer tutor}.
\newblock Speech Communication, 2011, v53, pages: 1115~1136.

\bibitem[\protect\citename{Barry and Shieber}2011]{Barry_Shieber:2011}
Barry Heather and Shieber Stuart M.
\newblock 2011.
\newblock {\em Recognizing Uncertainty in Speech}.
\newblock EURASIP Journal on Advances in Signal Processing, 2011.

\bibitem[\protect\citename{Forbes et al.}2009]{Forbes-Riley:2009}
Forbes-Riley, Kate and Litman, Diane.
\newblock 2011.
\newblock {\em Adapting to Student Uncertainty Improves Tutoring Dialogues}.
\newblock Proceedings of the 2009 conference on Artificial Intelligence in Education, 2009, pp. 33~40.

\bibitem[\protect\citename{Breazeal and Aryananda}2002]{Breazeal:2002}
Breazeal C. and Aryananda, L.
\newblock 2002.
\newblock {\em Recognition of affective communicative intent in robot-directed speech}.
\newblock Autonomous Robots 12 1, 2002. pp. 83�C104.

\bibitem[\protect\citename{Dellaert et al.}1996]{Dellaert:1996}
Dellaert, F., Polizin, t., and Waibel, A.
\newblock 1996.
\newblock {\em Recognizing Emotion in Speech}.
\newblock In Proc. Of ICSLP 1996, Philadelphia, PA, pp.1970-1973, 1996

\bibitem[\protect\citename{Lee et al.}2001]{Lee:2001}
Lee, C.M.; Narayanan, S.; Pieraccini, R.
\newblock 2001.
\newblock {\em Recognition of Negative Emotion in the Human Speech Signals}.
\newblock Workshop on Auto. Speech Recognition and Understanding, Dec 2001

\end{thebibliography}

\end{document}
