%
% File naaclhlt2012.tex
%

\documentclass[11pt,letterpaper]{article}
\usepackage{naaclhlt2012}
\usepackage{times}
\usepackage{latexsym}
\usepackage{amsmath}
\setlength\titlebox{6.5cm}    % Expanding the titlebox

\title{Tackling Dialog State Tracking Challenge}

\author{Wencan Luo\\
	    Department of Computer Science\\
	    University of Pittsburgh\\
	    PA 15260, USA\\
	    {\tt wencan@cs.pitt.edu}
	  }

\date{September 9, 2013}

\begin{document}
\maketitle
\begin{abstract}
We will propose a model to tackle the Dialog State Tracking Challenge \cite{Williams:2013}.

\end{abstract}

\section{Introduction}
In dialog systems, "state tracking" � sometimes also called "belief tracking" � refers to accurately estimating the user�s goal as a dialog progresses. Accurate state tracking is desirable because it provides robustness to errors in speech recognition, and helps reduce ambiguity inherent in language within a temporal process like dialog. Dialog state tracking is an important problem for both traditional uni-modal dialog systems, as well as speech-enabled multi-modal dialog systems on mobile devices, on tablet computers, and in automobiles \cite{Williams:2012}. 

The ``Dialog State Tracking Challenge"(DSTC) provided a good test bank for this task. This challenge 2013 has completed. 9 teams entered a total of 27 entries. Results has been shown at SigDial 2013.

The data is still public available \footnote{http://research.microsoft.com/en-us/events/dstc/. This link was broken a few days ago, but it is fixed up after I request}.

\section{Task Description}
DSTC data is taken from several different spoken dialog systems. All of them provide bus schedule information for Pittsburgh, Pennsylvania, USA \cite{Black:2011}. Different dialog system might have differen ASR, NLU and dialog control components. In this challenge, only 9 slots are evaluated: route, from.desc, from.neighborhood, from.monument, to.desc, to.neighborhood, to.monument, date, and time. The approximate number of distinck values for slots are shown in Table \ref{table:slot_num}. The number of values for each slot varies a lot.

The dialog tracker logs SLU N-best hypotheses for each user turn with confidence scores. As they claimed, the coverage of N-best hypotheses is good, so the challenge confines consideration of goals to slots and values that have been observed in an SLU output. The task of a dialog state tracker is to generate a set of observed slot and value pairs, with a score between 0 and 1. The sum of all scores should be 1.

For evaluation, there are 11 different metric, 4 test tests under 3 different schedules \cite{Williams:2013} for 9 slots.

\begin{table}[!htb] 
\centering
\begin{tabular}{c|c}
Slot name& number of values\\\hline
route&100\\
from.desc&500-10000\\
to.desc&500-10000\\
from.neighborhood&20-100\\
to.neighborhood&20-100\\
from.monument&50-500\\
to.monument&50-500\\
date.day&9\\
date.absmonth&12\\
date.absday&31\\
date.relweek&1\\
time.hour&12\\
time.minute&60\\
time.ampm&2\\
time.arriveleave&2\\
time.rel&1\\
\end{tabular}
\caption{Approximate number of distinct values for slots}
\label{table:slot_num}
\end{table}

\section{The Corpus}
The data is divided into 4 training sets and 4 test sets. They come from different sources. The basic statistical information for the corpus is shown in Table \ref{table:corpus}

\begin{table*}[!htb] 
\centering
\begin{tabular}{l|c|c|c|c}
Role&PM&ME&UI&ID\\\hline
Ratio&36.6\%&22.1\%&19.8\%&21.5\%\\
\end{tabular}

\caption{Dataset description}
\label{table:corpus}
\end{table*}

\section{Related Work}
\subsection{Overall Results}

\subsection{Methodalogy}


\section{Timeline}

\noindent \emph{Sep 09 - Sep 22}
\begin{itemize}
  \item survey the related work regarding role recognition
  \item understanding the data, know how to extract and use the data
\end{itemize}

\noindent \emph{Sep 23 -  Oct 20}
\begin{itemize}
  \item implement the method in \cite{Garg:2008} using the manual transcription, the lexical model will be used as the local model
  \item do Automatical Speech Recognition (ASR)
  \item run the local model on ASR results
\end{itemize}

\noindent \emph{Oct 21 - Nov 9}
\begin{itemize}
  \item implement ILP global model, using the manual speaker segmentation
\end{itemize}

\noindent \emph{Nov 10 - Dec 12}
\begin{itemize}
  \item propose a model without the manual speaker segmentation
  \item try other global model such as Bayes network, improved social network
\end{itemize}

\section*{Acknowledgments}

Do not number the acknowledgment section.

\begin{thebibliography}{}

\bibitem[\protect\citename{Williams et al.}2012]{Williams:2012}
Jason D Williams, Antoine Raux, Deepak Ramachandran, and Alan W Black.
\newblock 2012. 
\newblock {\em Dialog state tracking challenge handbook}. 
\newblock Technical report, Microsoft Research.

\bibitem[\protect\citename{Black et al.}2013]{Black:2011}
A. Black et al.
\newblock 2011. 
\newblock {\em Spoken dialog challenge 2010: Comparison of live and control test results}. 
\newblock In Proceedings of SIGDIAL.

\bibitem[\protect\citename{Williams et al.}2013]{Williams:2013}
Jason D. Williams, Antoine Raux, Deepak Ramachandran, and Alan Black.
\newblock 2013. 
\newblock {\em The Dialog State Tracking Challenge}. 
\newblock In Proceedings 14th Annual Meeting of the Special Interest Group on Discourse and Dialogue (SIGDIAL), Metz, France. 

\end{thebibliography}

\end{document}
