%
% File naaclhlt2012.tex
%

\documentclass[11pt,letterpaper]{article}
\usepackage{naaclhlt2012}
\usepackage{times}
\usepackage{latexsym}
\usepackage{amsmath}
\setlength\titlebox{6.5cm}    % Expanding the titlebox

\title{Modeling Conflicts in Multi-part Dialogue}

\author{Wencan Luo\\
	    Department of Computer Science\\
	    University of Pittsburgh\\
	    PA 15260, USA\\
	    {\tt wencan@cs.pitt.edu}
	  }

\date{September 9, 2013}

\begin{document}
\maketitle
\begin{abstract}
We will propose a model to predict conflicts for multi-part dialogue. Conflicts are disagreements between two or more people. It includes task, process, and relationship conflicts.

\end{abstract}

\section{Introduction}
A conflict is defined as disagreement between two or more people \cite{Paletz:2011}. In this paper, only short-term conflicts are considered. In other words, conflicts happen only in minutes but not last a couple of days and not even longer.

Conflicts can be categorized by task, process, and relationship types \cite{Jehn:1995,Jehn:1997}. Linguistic researchers have argued that task conflict, under certain circumstances, can be beneficial, particularly for innovation \cite{Jehn:1997,West:2002}, whereas relationship and process conflict should hurt performance \cite{Jehn:1997}.

Modeling conflicts in dialogues can be benefit to team development, dialog understanding and management, etc. Conflicts in general can also help credibility-based summarization \cite{Kaneko:2009}.

\section{Related Work}
Bracewell et al. \shortcite{Bracewell:2012} classified 11 social acts including agreement and disagreement on social medial based on gappy patterns with 50.4\% f-measure. A gappy pattern consists of one or more words in between which there can exist gaps, or wildcards, which match any word. Actually, the disagreement act in that paper is not exactly the same as conflict in our paper. The disagreement act is defined as ``statements a group member makes to indicate that he/she does not share the same
view about something another member has said or done". Conflicts here should also include the relationship conflict act defined as ``personal, heated disagreement between individuals".

Classifying agree/disagree opinions in conversational debates using Bayesian networks was presented in \cite{Galley:2004} based on adjacency pairs features. 
Agree/disagree classification is formulized as a max cut problem in \cite{Murakami:2010} for online debates.

Recently, Paletz et al. \shortcite{Paletz:2011} presented an extensive work on coding conflicts in natural multi-part dialogues.

\section{The Corpus}
We are going to use the Eng data \cite{Jang:2012,Friedberg:2012}, collected in University of Pittsburgh.
It is a collection of natural dialogues among teams of college undergraduates working on their semester-long product design projects.
The conversations involve 2-6 individuals. Most of the students were engineering majors (e.g., electrical, mechanical, and industrial), but some teams also had marketing students as members.

Among 45,687 utterances, 1,401 of them are annotated as conflicts. The conflict levels are also annotated, like ``low" and ``high". Their counts are shown in Table \ref{table:corpus_level}.
\begin{table}[!htb] 
\centering
\begin{tabular}{c|c|c}
Hi&Low&Unknown\\\hline
1149&197&55\\
\end{tabular}

\caption{number of conflicts and conflict level in the Eng corpus}
\label{table:corpus_level}
\end{table}

The distribution of types of conflicts is shown in Table \ref{table:corpus_type}.
\begin{table*}[!htb] 
\centering
\begin{tabular}{c|c|c|c|c}
type&Task&Process&Relationship&Off task&Off topic&Unknown\\\hline
\#&755&462&98&66&18&2\\
\end{tabular}

\caption{Distribution of types of conflict in the Eng corpus}
\label{table:corpus_type}
\end{table*}

Examples of conflicts are shown in Table \ref{table:corpus_examples}.
\begin{table*}[!htb] 
\centering
Speaker	Utterance	Conflict?	Level
1	You never know, they are-	1	hi
4	They are catching up, 	1	hi
4	But I doubt it.	1	hi
			
4	Can you crank out a couple things tonight?	0	
1	No	0	
4	Homework?	0	
1	No	1	low
			
4	Ok well the answer to the question is this is connected to that  	0	
4	and I'm not holding that	0	
4	and that's another person who's [?]	0	
1	For what	0	
4	You're insane - 	1	hi

\caption{Three examples of conflicts}
\label{table:corpus_examples}
\end{table*}

\section{Data Preprocessing}
The distribution of conflict and non-conflict is highly skewed (3\%). Thus, it might be better do resampling in order to balance it.
One method could  be is blocking introduced in \cite{Paletz:2011}. The idea is to divide the conversations into blocks. If a conflict happens, a block includes the conflict utterances, and also includes 25 utterances before it and 25 utterances after it\footnote{25 utterances are chosen because they are roughly one-minute.}. We can then randomly sample the rest to get blocks that have no conflict.

In this way, we can have half blocks that contain conflict and half blocks that not.

\section{Methodology}

\subsection{Classification Model}
Even we treate the conversations as blocks, we can treat each utterance independent with each other. Therefore, the task to identify conflict utterances is a binary-classification problem.

Features I will use include:

\begin{itemize}
	\item Ngram
	\item Word length
	\item POS
	\item Whether contains negative/positive words
	\item LIWC features
	\item Whether is the same speaker as the previous one
\end{itemize}

There are some other hand-made features in this corpus that can be used:
\begin{itemize}
	\item Whether this utterance is related to the project
	\item Uncertainty?
	\item Analogy occur?
\end{itemize}

\subsection{Sequence Labeling Model}
Usually, when a conflict happens, it takes a while to be resolved. Thus, it can also be treated as a sequence labeling problem. The task is to identify when a conflict begins and when it is over. We can use ``BIO" tags, where ``B" indicates the beginning of a conflict; ``I" means the inside of a conflict; ``O" means the ending of a conflict.

CRF \cite{Lafferty:2001} can be used for this model.

Besides the regular features, other features could be tried such as 
\begin{itemize}
	\item The previous tag
	\item Whether this utterance has same words as the previous one
\end{itemize}

\subsection{Event-Graph Model}
In this model, the assumption is that a conflict between two people should happen with an event. Let $e$ is an event, and there are two speakers $A$ and $B$. If $A$ agrees with $e$ but $B$ disagrees with $e$, then a conflict happens. In other words, speakers are connected by events and a speaker can either agree with an event or disagree it.

If we can construct this event graph, identify a conflict between two people will be pretty straightforward.

\section{Future Work}
Relying only on transcriptions might not be prefect for this problem. Firstly, sometimes, you cannot tell a conflict by the transcription but you can tell by their facial expression, body gestures, vocal changes, tone, etc. Secondly, conflicts dependent on the culture, which might be very hard to model.

\section{Timeline}

\noindent \emph{Sep 13 - Sep 22}
\begin{itemize}
  \item survey the related work regarding conflict/disagreement/agreement prediction
  \item resampling the data into blocks
\end{itemize}

\noindent \emph{Sep 23 -  Oct 20}
\begin{itemize}
  \item extract basic features
  \item implement the classification Model
\end{itemize}

\noindent \emph{Oct 21 - Nov 9}
\begin{itemize}
  \item implement CRF model
\end{itemize}

\noindent \emph{Nov 10 - Dec 12}
\begin{itemize}
  \item Implment the graph model
\end{itemize}

\section*{Acknowledgments}

Do not number the acknowledgment section.

\begin{thebibliography}{}
\bibitem[\protect\citename{Abbott et al.}2011]{Abbott:2011}
R. Abbott, M. Walker, P. Anand, J. E. F. Tree, R. Bowmani, and J. King.
\newblock 2011. 
\newblock {\em How can you say such things?!?: Recognizing disagreement in informal political argument}. 
\newblock In Proceedings of the Workshop on Languages in Social Media, LSM ’11, pages 2–11.

\bibitem[\protect\citename{Bracewell et al.}2012]{Bracewell:2012}
D. Bracewell, M. Tomlinson, H. Wang.
\newblock 2012. 
\newblock {\em Identification of social acts in dialogue}. 
\newblock In 24th International Conference on Computational Linguistics, COLING (2012)

\bibitem[\protect\citename{Friedberg et al.}2012]{Friedberg:2012}
H. Friedberg, D. Litman, and S. B. F. Paletz.
\newblock 2012. 
\newblock {\em Lexical Entrainment and Success in Student Engineering Groups}. 
\newblock Proceedings Fourth IEEE Workshop on Spoken Language Technology (SLT), pages 404-409, Miami, Florida, December.

\bibitem[\protect\citename{Galley et al.}2004]{Galley:2004}
M. Galley, K. McKeown, J. Hirschberg, and E. Shriberg.
\newblock 2004. 
\newblock {\em Identifying agreement and disagreement in conversational speech: use of bayesian networks to model pragmatic dependencies}. 
\newblock In ACL ’04: Proceedings of the 42nd Annual Meeting on Association for Computational Linguistics, pages 669–676, Morristown, NJ, USA. Association for Computational Linguistics.

\bibitem[\protect\citename{Lafferty et al.}2001]{Lafferty:2001}
J. Lafferty, A. McCallum, and F. Pereira.
\newblock 2001. 
\newblock {\em Conditional random fields: Probabilistic models for segmenting and labeling sequence data}. 
\newblock In Proc. of ICML, pp.282-289

\bibitem[\protect\citename{Murakami and Raymond}2010]{Murakami:2010}
A. Murakami and R. Raymond.
\newblock 2010. 
\newblock {\em Support or oppose?: classifying positions in online debates from reply activities and opinion expressions}. 
\newblock In Proceedings of Coling 2010: Poster Volume.

\bibitem[\protect\citename{Jang and Schunn}2012]{Jang:2012}
Jooyoung Jang and Christian Schunn.
\newblock 2012.
\newblock {\em Physical design tools support and hinder innovative engineering design}.
\newblock Journal of Mechanical Design, vol. 134, no. 4, pp. 041001-1-041001-9.

\bibitem[\protect\citename{Jehn}1995]{Jehn:1995}
K. A. Jehn.
\newblock 1995. 
\newblock {\em A multimethod examination of the benefits and detriments of intragroup conflict}. 
\newblock Administrative Science Quarterly, 40, 256–282.

\bibitem[\protect\citename{Jehn}1997]{Jehn:1997}
K. A. Jehn.
\newblock 1997. 
\newblock {\em A qualitative analysis of conflict types and dimensions in organizational groups}. 
\newblock Administrative Science Quarterly, 42, 530–557.

\bibitem[\protect\citename{Kaneko et al.}2009]{Kaneko:2009}
K. Kaneko, H. Shibuki, M. Nakano, R. Miyazaki, M. Ishioroshi, and T. Mori.
\newblock 2009. 
\newblock {\em Mediatory summary generation: Summary-passage extraction for information credibility on the web}. 
\newblock In Proceedings of the 23rd Pacific Asia Conference on Language, Information and Computation (PACLIC 23), pages 240--249.

\bibitem[\protect\citename{Paletz et al.}2011]{Paletz:2011}
S. B. F. Paletz, C. D. Schunn, and K. H. Kim
\newblock 2011. 
\newblock {\em Conflict under the microscope: Micro-conflicts in naturalistic team discussions}. 
\newblock Negotiation and Conflict Management Research, 4, 314-351.

\bibitem[\protect\citename{West}2002]{West:2002}
M. A. West.
\newblock 2002. 
\newblock {\em Sparkling fountains or stagnant ponds: An integrative model of creativity and innovation implementation in work groups}. 
\newblock Applied Psychology: An International Review, 51, 355–424.

\end{thebibliography}

\end{document}
