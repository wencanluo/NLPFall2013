%
% File naaclhlt2012.tex
%

\documentclass[11pt,letterpaper]{article}
\usepackage{naaclhlt2012}
\usepackage{times}
\usepackage{latexsym}
\usepackage{amsmath}
\setlength\titlebox{6.5cm}    % Expanding the titlebox

\title{Modeling Conflicts in Multi-part Dialogue}

\author{Wencan Luo\\
	    Department of Computer Science\\
	    University of Pittsburgh\\
	    PA 15260, USA\\
	    {\tt wencan@cs.pitt.edu}
	  }

\date{September 9, 2013}

\begin{document}
\maketitle
\begin{abstract}
We will propose a model to predict conflicts for multi-part dialogue.

\end{abstract}

\section{Introduction}
Conflict is defined as disagreement between two or more people \cite{Paletz:2011}. In this paper, only short-term conflicts are considered. In other words, conflicts happen only in minutes but not a couple of days and not even longer.

The conflicts can be categorized by task, process, and relationship type \cite{Jehn:1995,Jehn:1997}. Linguistic researchers have argued that task conflict, under certain circumstances, can be beneficial, particularly for innovation \cite{Jehn:1997,West:2002}, whereas relationship and process conflict should hurt performance \cite{Jehn:1997}.

Thus, modeling conflicts in dialogues can be benefit to team development, 

\section{Related Work}

\section{The Corpus}

We are going to use the Eng data \cite{Jang:2012,Friedberg:2012}, collected in University of Pittsburgh.
It is a collection of natural dialogues among teams of college undergraduates working on their semester-long product design projects.
The conversations involve 2-6 individuals. Most of the students were engineering majors (e.g., electrical, mechanical, and industrial), but some teams also had marketing students as members.


\section{Data Preprocessing}

\section{Methodology}

\subsection{Classification Model}
\begin{itemize}

\end{itemize}

\subsection{Sequence Labeling Model}

\subsection{Event-Graph Model}

\section{Future Work}
Predict conflict on speech.

\section{Timeline}

\noindent \emph{Sep 09 - Sep 22}
\begin{itemize}
  \item survey the related work regarding role recognition
  \item understanding the data, know how to extract and use the data
\end{itemize}

\noindent \emph{Sep 23 -  Oct 20}
\begin{itemize}
  \item implement the method in \cite{Garg:2008} using the manual transcription, the lexical model will be used as the local model
  \item do Automatical Speech Recognition (ASR)
  \item run the local model on ASR results
\end{itemize}

\noindent \emph{Oct 21 - Nov 9}
\begin{itemize}
  \item implement ILP global model, using the manual speaker segmentation
\end{itemize}

\noindent \emph{Nov 10 - Dec 12}
\begin{itemize}
  \item propose a model without the manual speaker segmentation
  \item try other global model such as Bayes network, improved social network
\end{itemize}

\section*{Acknowledgments}

Do not number the acknowledgment section.

\begin{thebibliography}{}

\bibitem[\protect\citename{Paletz at el.}2011]{Paletz:2011}
S. B. F. Paletz, C. D. Schunn, and K. H. Kim
\newblock 2011. 
\newblock {\em Conflict under the microscope: Micro-conflicts in naturalistic team discussions}. 
\newblock Negotiation and Conflict Management Research, 4, 314-351.

\bibitem[\protect\citename{Friedberg et al.}1990]{Friedberg:2012}
H. Friedberg, D. Litman, and S. B. F. Paletz.
\newblock 2012. 
\newblock {\em Lexical Entrainment and Success in Student Engineering Groups}. 
\newblock Proceedings Fourth IEEE Workshop on Spoken Language Technology (SLT), pages 404-409, Miami, Florida, December.

\bibitem[\protect\citename{Jehn}1995]{Jehn:1995}
Jehn, K. A.
\newblock 1995. 
\newblock {\em A multimethod examination of the benefits and detriments of intragroup conflict}. 
\newblock Administrative Science Quarterly, 40, 256–282.

\bibitem[\protect\citename{Jehn}1997]{Jehn:1997}
Jehn, K. A.
\newblock 1997. 
\newblock {\em A qualitative analysis of conflict types and dimensions in organizational groups}. 
\newblock Administrative Science Quarterly, 42, 530–557.

\bibitem[\protect\citename{West}2002]{West:2002}
West, M. A.
\newblock 2002. 
\newblock {\em Sparkling fountains or stagnant ponds: An integrative model of creativity and innovation implementation in work groups}. 
\newblock Applied Psychology: An International Review, 51, 355–424.

\end{thebibliography}

\end{document}
