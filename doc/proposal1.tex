%
% File naaclhlt2012.tex
%

\documentclass[11pt,letterpaper]{article}
\usepackage{naaclhlt2012}
\usepackage{times}
\usepackage{latexsym}
\setlength\titlebox{6.5cm}    % Expanding the titlebox

\title{Role Recognition for Multi-part Dialogue: A Combined Global and Local Approach}

\author{Wencan Luo\\
	    Department of Computer Science\\
	    University of Pittsburgh\\
	    PA 15260, USA\\
	    {\tt wencan@cs.pitt.edu}
	  }

\date{September 3, 2013}

\begin{document}
\maketitle
\begin{abstract}
  
\end{abstract}

\section{Introduction}

``People do not interact with one another as anonymous beings. They come together in the context of specific environments and with specific purposes" \cite{Tischler:1990}. As an example, people involved in multi-part dialogue usually play certain roles. For example, Radio Broadcasts

Speaker role is an important cue to the structure dialogue. It can be benefit to role-based summerization \cite{Vinciarelli:2006}, semantical coherent segmentation, information retrieval \cite{Weng:2007,Knapp:1972}, etc.

Role recognition is the task of automatically recognizing roles of participants in an interaction
recording. The goal is to assign to every participant in the recording of an interaction
(usually and audio recording or video recording) a role \cite{Salamin:2013}.

In this paper, we will propose a new method for role recognition, which combines both the global and local constraints. There are two intuitions behind: firstly, during a converstion, the role of a participant doesn't change; secondly, the defined roles should be taken evenly among the participants. Take a two-person interview for example. Firstly, a interviewer is always interviewing during the conversation; Secondly, if one is the interviewer and the other must be the interviewee.

\section{Related Work}

Barzilay et al. \shortcite{Barzilay:2000} exploited the lexical information (from ASR transcriptions) to identify 3 type of roles: Anchor, Journalist, Guest speakers in news broadcast.

Garg et al. \shortcite{Garg:2008} identified four predefined roles for multi-part meetings. It combined lexical features and social network (SNA) based on linear model. 
They also extracted features from the social network \cite{Salamin:2009}. 
Later, they proposed a graph model based on purely nonverbal vocal behavioral cues, including who talks when and how much (turn-taking behavior), and statistical properties of pitch, formants, energy and speaking rate (prosodic behavior)\cite{Salamin:2010}.

Dynamic Bayesian Networks \cite{Yaman:2010} is also used in role recognition.

\section{The Corpus}
The corpus I will use is the AMI
corpus \cite{McCowan:2005}, as same as one used in \cite{Garg:2008,Salamin:2009,Salamin:2010}.

The AMI corpus a collection of 138 meeting recordings for a total
of 45 hours and 38 minutes of material in a simulated environment. In each meeting, four partici-
pants play the following roles: the Project Man-
ager (PM), the Marketing Expert (ME), the User Interface
Expert (UI), and the Industrial Designer (ID). Each par-
ticipant plays a different role, and all roles are represented
in each meeting. The same person can play different roles
in different meetings, and the ratio of meeting time that
each role accounts for, on average, is reported in Table \ref{table:corpus}.

Currently, the state-of-art accuracy is 67.9\% on the AMI meeting corpus \cite{Garg:2008, Salamin:2013} by combining lexical information and social network analysis.

\begin{table}[!htb] 
\centering 
\begin{tabular}{l|c|c|c|c}
Role&PM&ME&UI&ID\\\hline
Ratio&36.6\%&22.1\%&19.8\%&21.5\%\\
\end{tabular}
 
\caption{Role distribution.} 
\label{table:corpus} 
\end{table} 

\section{Methodology}


\section{Timeline}

\noindent \emph{Jan 12 - Jan 20}
\begin{itemize}
  \item understanding the data, know how to extract and use the data
  \item survey the related work regarding uncertainty prediction based on text
\end{itemize}

\noindent \emph{Jan 21 - Feb 8}
\begin{itemize}
  \item coding the rule-based methods
  \item extract the unigram, bigram and trigram features
  \item prelim results based this simple lexical features using Weka
  \item present the prelim results
\end{itemize}

\noindent \emph{Feb 9 - March 9}
\begin{itemize}
  \item based on the suggestions, improve the model
  \item develop more lexical features
  \item extract other features, such as syntax, semantic
\end{itemize}

\noindent \emph{March 10 - April 20}
\begin{itemize}
  \item improve the model based on new features
  \item anaylze the confusion matrix of the model and improve the model based on the results
  \item write the workshop-like paper
\end{itemize}

\section*{Acknowledgments}

Do not number the acknowledgment section.

\begin{thebibliography}{}

\bibitem[\protect\citename{Tischler}1990]{Tischler:1990}
H. Tischler.
\newblock 1990. 
\newblock {\em Introduction to Sociology}. 
\newblock Harcourt Brace College Publishers.

\bibitem[\protect\citename{Vinciarelli}2006]{Vinciarelli:2006}
A. Vinciarelli
\newblock 2006. 
\newblock {\em Sociometry based multiparty audio recordings summarization}. 
\newblock in 18th International Conference on Pattern Recognition, vol. 2. IEEE, 2006, pp. 1154�1157.

\bibitem[\protect\citename{Weng at el.}2007]{Weng:2007}
C. Weng, W. Chu, and J. Wu.
\newblock 2007. 
\newblock {\em Movie analysis based on roles' social network}. 
\newblock in IEEE International Conference on Multimedia and Expo, pp. 1403�1406.

\bibitem[\protect\citename{Knapp and Hall}1972]{Knapp:1972}
 M. Knapp and J. Hall.
\newblock 1972. 
\newblock {\em Nonverbal Communication in Human Interaction}. 
\newblock Harcourt Brace College Publishers.

\bibitem[\protect\citename{Barzilay et al.}2000]{Barzilay:2000}
R. Barzilay, M. Colins, J. Hirschberg, and S. Whittaker. 
\newblock 2000. 
\newblock {\em The rules behind roles: Identifying speaker role in radio broadcasts}. 
\newblock Proc. AAAI Conference on Artificial Intelligence \& Conference on Innovative Applications of Artificial Intelligence, 679-684. AAAI Press/MIT Press.

\bibitem[\protect\citename{Garg et al.}2008]{Garg:2008}
N. Garg, S. Favre, H. Salamin, D. Hakkani-Tur, and A. Vinciarelli. 
\newblock 2008. 
\newblock {\em Role recognition for meeting participants:an approach based on lexical information and social network analysis}.
\newblock Proceedings ACM International Conference on Multimedia, 693-696.

\bibitem[\protect\citename{Salamin et al.}2009]{Salamin:2009}
H. Salamin, S. Favre, and A. Vinciarelli. 
\newblock 2009. 
\newblock {\em Automatic role recognition in multiparty recordings: Using social affiliation networks for feature extraction}.
\newblock IEEE Trans. Multimedia, vol. 11, no. 7, pp. 1373�1380

\bibitem[\protect\citename{Salamin et al.}2010]{Salamin:2010}
H. Salamin , A. Vinciarelli , K. Truong and G. Mohammadi.
\newblock 2010. 
\newblock {\em Automatic role recognition based on conversational and prosodic behaviour}.
\newblock Proceedings of the international conference on Multimedia, October 25-29, 2010, Firenze, Italy 

\bibitem[\protect\citename{Salamin and Vinciarelli}2012]{Salamin:2012}
H. Salamin and A. Vinciarelli.
\newblock 2012. 
\newblock {\em Automatic role recognition in multiparty conversations: An approach based on turn organization, prosody and conditional random fields}.
\newblock IEEE Trans. Multimedia, vol. 14, no.2, pp. 338�345, 2012.

\bibitem[\protect\citename{Salamin}2013]{Salamin:2013}
H. Salamin.
\newblock 2013. 
\newblock {\em Automatic role recognition}.
\newblock PhD thesis, University of Glasgow.

\bibitem[\protect\citename{Yaman et al.}2010]{Yaman:2010}
S. Yaman, D. Hakkani-Tu\'{r}, G. Tu\'{r}.
\newblock 2010.
\newblock {\em Social role discovery from spoken language using Dynamic Bayesian Networks}. 
\newblock Proc. of Interspeech, 2010. 

\bibitem[\protect\citename{McCowan et al.}2005]{McCowan:2005}
I. McCowan, J. Carletta, W. Kraaij, S. Ashby, S. Bourban, M. Flynn, M. Guillemot, T. Hain, J. Kadlec, V. Karaiskos, M. 
Kronenthal, G. Lathoud, M. Lincoln, A. Lisowska, W. Post, D. Reidsma, and P. Wellner. 
\newblock 2005.
\newblock {\em The ami meeting corpus}. 
\newblock In Proceedings of the 5th International Conference on Methods and Techniques in Behavioral Research.

\end{thebibliography}

\end{document}
